\chapter{Metodologia}

Esse trabalho procura encontrar um método automatizado de detecção de defeitos de fabricação em placas de circuito impresso, buscando uma solução eficiente e precisa. Para isso, realizou-se um estudo de caso com os seguintes tipos de pesquisa: aplicada, qualitativa do tipo exploratória e bibliográfica.

A pesquisa, segundo \citeonline{ref:Gil}, pode ser definida como ``o processo formal e sistemático do processo científico [cujo] objetivo [$\cdots$] é descobrir respostas para problemas mediante o emprego de procedimentos científicos''. Considerando isso, buscou-se realizar uma pesquisa aplicada a fim de investigar um dos principais problemas deste trabalho: a detecção de defeitos em placas de circuito impresso.

A pesquisa qualitativa de caráter exploratório pressupõe a interpretação do pesquisador na análise dos dados. Segundo \citeonline{ref:Gil}, a pesquisa exploratória ``tem como finalidade desenvolver, esclarecer e modificar conceitos e ideias, tendo em vista a formulação de problemas mais precisos ou hipóteses pesquisáveis para estudos posteriores''. Nesse trabalho, o treinamento de uma rede neural para a detecção de defeitos de fabricação em placas de circuito impresso tem como partida a análise de resultados obtidos em trabalhos anteriores, de forma a aprimorar esses resultados e expor melhoras para trabalhos futuros.

Para a construção do referencial teórico, uma pesquisa bibliográfica foi realizada. Esse tipo de pesquisa é desenvolvida a partir de material já elaborado, constituído principalmente de livros e artigos científicos \cite{ref:Gil}.
