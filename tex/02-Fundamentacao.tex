\chapter{Fundamentação Teórica} \label{cap:fund}

Nesse capítulo serão apresentados os principais conceitos necessários para a compreensão desse trabalho.
%Nessa seção serão desenvolvidos conceitos que envolvem a Inteligência Artificial e suas sub-áreas.
\section{Inteligência Artificial} \label{cap:fund-ia}

Apesar de ter chamado atenção nos últimos anos com a quantidade enorme de dados adquiridos e processados por grandes corporações como Google, Facebook, Amazon, e Apple \cite{ref:Lawless-Mittu-Sofge}, o termo `Inteligência Artificial' não é tão atual assim. Ele foi utilizado pela primeira vez em 1955 por John McCarthy \cite{ref:Cohen}, que conduziu no ano seguinte um workshop cuja premissa central da proposta considera que o comportamento humano inteligente consiste em processos que podem ser formalizados e reproduzidos por uma máquina \cite{ref:Harvard-AI}. %conjectura

Um dos objetivos da Inteligência Artificial, de acordo com \citeonline{ref:Mitchell-Michalski-Carbonell}, é fazer com que computadores realizem tarefas mais inteligentes de forma com que não haja necessidade dos seres humanos executá-las. Entretanto, um dos grandes desafios da área de IA atualmente é a execução de atividades consideradas simples e corriqueiras para pessoas, como reconhecimento de objetos e fala \cite{ref:Goodfellow-Bengio-Courville}.



%Segundo \citeonline{ref:Goodfellow-Bengio-Courville}, a Inteligência Artificial (IA) em seus primórdios foi muito utilizada para resolução de problemas matemáticos que são complexos para seres humanos, porém simples para computadores. Porém, ainda segundo \citeonline{ref:Goodfellow-Bengio-Courville}, o verdadeiro desafio da IA atualmente é resolver problemas que na verdade são tarefas simples para pessoas mas complexas para computadores, como fazer o reconhecimento de fala ou rostos em fotos.

%Porém apenas recentemente houveram avanços em pesquisas permitindo com que as habilidades de computadores fossem igualadas as do seres humanos em tarefas mais corriqueiras, como reconhecimento de objetos e de fala.

%Tarefas mais abstratas que podem ser descritas por uma lista de regras formais como jogar xadrez, ao contrário de tarefas mais corriqueiras como reconhecimento de objetos, onde avanços nas pesquisas foram obtidos apenas recentemente permitindo igualar habilidades humanas com as de computadores.


\section{Aprendizagem Profunda} \label{cap:fund-aprendizagem}

\section{Redes Neurais Profundas} \label{cap:fund-redes_profundas}

\section{Redes Neurais Convolucionais} \label{cap:fund-redes_convolucionais}

\section{Frameworks e Bibliotecas} \label{cap:fund-frameworks}

    \subsection{OpenCV} \label{cap:fund-frameworks-opencv}

    \subsection{Darknet} \label{cap:fund-frameworks-darknet}

    \subsection{Flask} \label{cap:fund-frameworks-flask}


\begin{comment}


\end{comment}
