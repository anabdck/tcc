%%%%%%%%%%%%%% Como usar o pacote acronym
% \ac{acronimo} -- Na primeira vez que for citado o acronimo, o nome completo irá aparecer
%                  seguido do acronimo entre parênteses. Na proxima vez somente o acronimo
%                  irá aparecer. Se usou a opção footnote no pacote, entao o nome por extenso
%                  irá aparecer aparecer no rodapé
%
% \acf{acronimo} -- Para aparecer com nome completo + acronimo
% \acs{acronimo} -- Para aparecer somente o acronimo
% \acl{acronimo} -- Nome por extenso somente, sem o acronimo
% \acp{acronimo} -- igual o \ac mas deixando no plural com S (ingles)
% \acfp{acronimo}--
% \acsp{acronimo}--
% \aclp{acronimo}--
%\begin{acronym}
%	\acro{ABNT}{Associação Brasileira de Normas Técnicas}
%	\acro{abnTeX}{ABsurdas Normas para TeX}
%	\acro{AC}{Autoridade Certificadora}
%	\acro{AES}{\textit{Advanced Encryption Standard}}
%	\acro{TLS}{\textit{Transport Layer Security}}
%	\acro{TPC}{Terceira Parte Confiável}
%\end{acronym}

% ----------------------------------------------------------
% LISTA DE ABREVIATURAS E SIGLAS
% ----------------------------------------------------------
\begin{siglas}
    \item[SMPS] \textit{Switched Mode Power Supply} - Fonte chaveada
    \item[CC] Corrente Contínua
    \item[CA] Corrente Alternada
    \item[EMC] \textit{Electromagnetic Compatibility} - Compatibilidade Eletromagnética
    \item[EMI] \textit{Electromagnetic Interference} - Interferência Eletromagnética
    \item[EMS] \textit{Electromagnetic Suscptibility} - Susceptibilidade Eletromagnética
    \item[CISPR] \textit{Comité International Special des Perturbations Radioélectriques} - Comitê Especial Internacional de Rádio Interferência
    \item[ESD] \textit{Electrostatic Discharge} - Descarga Eletrostática
   
   \item[PCI] Placa de Circuito Impresso
   
   \item[EMP] \textit{Electromagnetic Pulse} - Pulso Eletromagnético
   \item[FCC] \textit{Federal Communication Commission}
   
   \item[IEC] \textit{International Electrotechnical Commission}
   \item[LISN] \textit{Line Impedance Stabilization Network}
   \item[NFP] \textit{Near-Field Probe} - Sonda de Campo Próximo
   \item[DUT] \textit{Device Under Test} - Dispositivo Sob Teste
   \item[OATS] \textit{Open-Area Test Site} - Local de Teste de Área Aberta
   \item[SAC] \textit{Semi Anechoic Chamber} - Câmara Semi Anecoíca
   \item[CI] Circuito Integrado
   \item[IFSC] Instituto Federal de Educação Ciência e Tecnologia de Santa Catarina
   \item[VDC] \textit{Voltage Direct Current} - Tensão Contínua
   \item[VAC] \textit{Voltage Alternating Current} - Tensão Alternada
   \item[ABNT] Associação Brasileira de Normas Técnicas
   \item[abnTex] Normas para \LaTeX
   \item[PTH] Through-Hole Technology
   \item[SMD] Surface-Mount Device
   \item[ESR] Equivalent Series Resistance
\end{siglas}


% ----------------------------------------------------------
% LISTA DE SIMBOLOS
% ----------------------------------------------------------
\begin{simbolos}
   \item[$V$] Volts - Unidade de potencial elétrico
   \item[$A$] Ampere - Unidade de Corrente Elétrica
   \item[$\Omega$] Ohms - Unidade de resistência elétrica
   \item[$F$] Faradays - Unidade de capacitância elétrica
   \item[$H$] Henry - Unidade de indutância elétrica
   \item[$W$] Watt's - Unidade de potência elétrica
   \item[$Hz$] Hertz - Unidade de Frequência (Ciclos por Segundo)
   \item[$VA$] Volt-Ampere - Unidade de potência elétrica
   \item[$dB\mu V$] Decibel microVolt - Decibéis relativos a um microVolt
   \item[$dB\mu V/m$] Decibel microVolt por metro - Decibéis relativos a um microVolt por metro
   \item[$mm$] Milimetros - Unidade de comprimento (1 metro divido por Mil)
   \item[$mm^{2}$] Milimetros Quadrados - Unidade de área
   \item[$\,^{\circ}\mathrm{C}$] Grau Celcius - Unidade de temperatura
\end{simbolos}


%%%%%%%%%%%%%%%%%%%%%%%%%%%%%%%%%%%%%%%%%%%%%%%%%%%%%%%
\marcador{nota}{Descrever a disposição do trabalho após o conteúdo estar definido}

Meu \abreviatura*{TCC}{Trabalho de Conclusão de Curso} está, segundo a \abreviatura.{ABNT}{Associação Brasileira de Normas Técnicas}, usando \abreviatura&{SMPS}{Switched Mode Power Supply}{fontes chaveadas}. 

\abreviatura'{PCB}{Printed Circuit Board}{Placa de Circuito Impresso}

\abreviatura{TCC}{Trabalho de Conclusão de Curso}

Valor de 3 \simbolo*{$\Omega$}{Ohms - Unidade de resistência elétrica} e 
7 \si{\volt},
valor de 5 \si{\micro\simbolo*{\si{\farad}}{Faradays - Unidade de capacitância elétrica}} e
4 \simbolo*{\si{\volt}}{Volts - Unidade de potencial elétrico}
ou \SI{5}{\milli\simbolo*{\si{\ampere}}{Ampere - Unidade de corrente elétrica}}.

\simbolo{\si{\henry}}{Henry - Unidade de indutância elétrica}
\abreviatura{TCC}{Trabalho de Conclusão de Curso}