\chapter{Introdução}
Placas de circuito impresso são de longe o método mais utilizado para o projeto de eletrônicos modernos. Elas consistem em um sanduíche de uma ou mais camadas de cobre intercaladas com uma ou mais camadas de material isolante \cite{ref:Zumbahlen} e servem de suporte para os componentes eletrônicos responsáveis pelo funcionamento de um circuito eletrônico.

À medida em novas tecnologias são desenvolvidas, as placas de circuito impresso estão se tornando cada vez mais sofisticadas e delicadas \cite{ref:Hu-Wang}, de modo que a detecção de incertezas, tolerâncias, defeitos e erros de posição relativa associados ao processo de fabricação \cite{ref:Leta-Feliciano-Martins} deve ser feita com mais cautela a fim de garantir o funcionamento do produto final.

Dessa forma, automatizar a inspeção de defeitos se tornou essencial para aprimorar a qualidade do processo de fabricação, já que técnicas de medição por visão computacional apresentam melhor regularidade, precisão e repetibilidade quando comparada a inspeção humana que, além da imprecisão e não-repetibilidade, está sujeita a subjetividade, fadiga e lentidão e está, ainda, associada a um alto custo \cite{ref:Leta-Feliciano-Martins}.

A inspeção ótica automatizada tem sido amplamente utilizada para detectar defeitos durante o processo de fabricação de uma PCB \cite{ref:Chin-Harlow}. Conforme a evolução dessa tecnologia, três principais métodos de detecção tem se destacado: métodos comparativos, não-referenciais e híbridos \cite{ref:Wu-Wang-Liu}. O método comparativo, mais utilizado entre eles, está susceptível à interferência da iluminação e ruídos externos, além de necessitar mecanismos de alinhamento precisos para realização da comparação \cite{ref:Hu-Wang}. Com o avanço nos estudos da área de Aprendizado Profundo, o método não-referencial utilizando Redes Neurais tem mostrado ótimo desempenho para a detecção de objetos, podendo ser utilizado para a detecção de defeitos em placas de circuito impresso com resultados de alta confiabilidade.

Sendo assim, esse trabalho propõe o uso de técnicas de Aprendizado Profundo para a detecção de defeitos em placas de circuito impresso, como maneira de aprimorar o processo de fabricação e melhorar a qualidade das mesmas.

\section{Justificativa}
O projeto de placas de circuito impresso tem se tornado cada vez mais complexo com o desenvolvimento de novas tecnologias de componentes eletrônicos e com a busca pela minimização dos produtos finais. Além desse fator, existe o alto custo atrelado a todo o processo de fabricação de cada placa, desde o projeto até a finalização, de forma que é fundamental que eventuais defeitos que possam ser inseridos nas linhas de fabricação das placas devem ser detectados de maneira confiável e rápida.

Dessa forma, automatizar o processo de detecção de defeitos utilizando técnicas de Aprendizado Profundo e uso de Redes Neurais permite o aprimoramento da qualidade do processo de fabricação e contribui com a funcionalidade e baixo custo do produto final.

\section{Definição do Problema}



\section{Objetivo Geral}

\section{Objetivos Específicos}
