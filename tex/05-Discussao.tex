\chapter{Discussão}

A rede neural treinada possui resultados satisfatórios para o \textit{dataset} utilizado, conforme apresentado nas Tabelas \ref{tab:resultados-metricas} e \ref{tab:resultados-metricas-total}. A \autoref{tab:comparacao-hripcb} apresenta a métrica de mAP para diferentes estruturas de redes neurais treinadas com o mesmo conjunto de dados com a finalidade de comparação com os resultados obtidos utilizando a YOLO versão quatro.

\begin{table}[!h]
  \begin{center}
    \caption{Comparação da utilização do conjunto de dados HRIPCB para o reconhecimento de defeitos em placas de circuito impresso para diferentes modelos de Redes Neurais.}
    \label{tab:comparacao-hripcb}
      \begin{tabular}{ccc}
      \toprule
      \textbf{Modelo} & \textbf{Referência} & \textbf{mAP} \\
      \midrule \midrule
      Faster R-CNN / VGG-16       & \apud{ref:Ren-et-al}{ref:Ding-et-al}    & 58,57\% \\
      Faster R-CNN / ResNet-101   & \apud{ref:Ren-et-al}{ref:Ding-et-al}    & 94,27\% \\
      FPN                         & \apud{ref:Lin-et-al-2}{ref:Ding-et-al}  & 92,23\% \\
      Faster R-CNN fine-tuned     & \cite{ref:Ding-et-al}                   & 96,44\% \\
      TDD-Net                     & \cite{ref:Ding-et-al}                   & 98,90\% \\
      YOLO versão 4               & proposto                                & 98,44\% \\
      \bottomrule
      \end{tabular}
    \indentedfont[0.96\textwidth]{Elaboração própria (2021)}
  \end{center}
\end{table}

Dentre as redes neurais apresentadas na \autoref{tab:comparacao-hripcb}, apenas a TDD-Net proposta por \citeonline{ref:Ding-et-al} apresentou um melhor desempenho do que a YOLO em sua quarta versão, cujos resultados foram apontados no \autoref{cap:treinamento}. Esse resultado pode melhorar com o treinamento de mais \textit{batches} da rede neural, ao invés de apenas dois mil e quinhentos.

Para imagens que não são do \textit{dataset} porém, os resultados apresentaram um desempenho não satisfatório, conforme mostram as Figuras \ref{fig:resultados-predicao-ruim-1} e \ref{fig:resultados-predicao-ruim-2}. Isso é consequência do conjunto de dados escolhido, que possui apenas imagens de placas de circuito impresso dentro de um padrão, conforme as imagens do \autoref{apendice:hripcb-pcbs}, e na indústria existem diversos tipos de placas de circuito impresso, com cores e tamanhos distintos.

%ainda falta finalizar
