\chapter{Considerações Finais}
Com o treinamento de uma rede neural foi possível fazer a detecção e localização de defeitos de fabricação em placas de circuito impresso utilizando um método não referencial. Considerando as características dos defeitos nas placas, o uso da rede neural YOLO em sua quarta versão foi considerado, possibilitando a obtenção de resultados satisfatórios para essa tarefa, em comparação com outras arquiteturas apresentadas em trabalhos anteriores (\autoref{tab:comparacao-hripcb}).

Para melhorar a detecção em placas de circuito impresso que são diferentes do padrão estabelecido pelo \textit{dataset} utilizado, recomenda-se para trabalhos futuros o treinamento da rede neural com placas que possuem características diferentes, como as apresentadas nas Figuras \ref{fig:resultados-predicao-ruim-1} e \ref{fig:resultados-predicao-ruim-2}, aumentando o \textit{dataset} utilizado.  Para acelerar o treinamento da rede neural para isso, pode-se utilizar a técnica de transferência de aprendizado discutida na \autoref{cap:treinamento-treinamento} com os pesos da rede neural obtidos como resultado desse trabalho, arquivo \url{yolov4_custom.weights} disponível em \url{https://github.com/anabdck/darknet}.

Além disso, como sugestão para trabalhos futuros, pode-se considerar uma comparação do desempenho da rede neural utilizando técnicas de pré-processamento de imagem, como binarização e operações morfológicas, sobre o \textit{dataset} utilizado, como forma de padronizar as imagens a serem treinadas.

A elaboração desse trabalho foi de grande relevância para a formação do aluno, possibilitando um complemento ao aprendizado além das unidades curriculares vistas no decorrer do curso de Engenharia Eletrônica. Além disso, esse trabalho pode servir como referência para trabalhos futuros, agregando conhecimento para a área de aprendizado profundo e redes neurais.
